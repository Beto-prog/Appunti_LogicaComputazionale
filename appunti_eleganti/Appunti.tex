\documentclass [12pt, letterpaper]{article}
\usepackage{amsmath}


\begin{document}
	\title{Appunti di Logica computazionale}
	\author{Rovesti Alberto}
	\date{2024}
	
	\maketitle
	
	Lo scopo di questa materia è quello di formalizzare il processo logico definendo verità oggettive in modo da creare ragionamenti/ conclusioni non ambigui
	
	
	
	\section{Concetti base}
	\begin{itemize}
		\item \textbf{Percezione} :
		
		Attraverso i nostri sensi noi non percepiamo la totalità del mondo e intlre possiamo percepirlo distorto Es: illusioni ottiche, giochi di luce 
		
		\item \textbf{Concettualizzazione} :
		
		Una volta osservato il mondo ci costruiamo in mente dei concetti che descrivono questo mondo: es:
		ho visto 100 sedie , se vedio un altra sedia anche diversa da tutte quelle viste prima la riconosco perchè ho una rappresentazione concettuale di cosa una sedia sia. Questo porta ad errori perchè come noi concepiamo le cose può essere influenzato dal nostro linguaggio (es: esperimento schianto macchine - a delle persone è stato fatto vedere un incidente e dopo gli hanno chiesto cosa fosse successo all'auto, le persone che conducevano gli esperimenti hanno usato diverse parole con diverse paersone per porre la domanda (le macchine si sono colpite, si sono scontrate, si sono toccate, si sono schiantate) in base alla gravità del termine utilizzato la risposta era conseguentemente catastrofica )
		
		\item \textbf{Rappresentazione} :
		
		Quando vogliamo descrivere qualcosa a qualcuno commettiamo sempre degli errori di rappresentazione. 
		Esempio: ci ricordiamo male, abbiamo dei baias, memorie parziale e più in generale tralasciamo un infinità di dettagli (che non possiamo comunicare per via del fatto che la nostra memoria è limitata e per un vincolo linguistico)
		
		\item \textbf{Ragionamento} :
		
		Partendo che i nostri pensieri sono basati su questi 3 concetti precedenti la nostra mente ragiona ed elabora e può portare a conclusioni sbagliate 
		
	\end{itemize}
	
	In conclusione il punto chiave è che tutti noi percepiamo il mondo in maniera diversa (semantic gap) e quindi dobbiamo trovare/costruire un sistema che ci permetta di esprimere in maniera inequivocabile se qualcosa sia VERO o FALSO per poter poi costruire concetti più complessi
	
	\section{Elementi fondanti}
	
	Prima di procedere dobbiamo chiarire 2 distinti elementi che sono fondamentali per la costruzione della logica:
	le \textbf{entità (entity)}, i \textbf{fatti (facts)} e le \textbf{Rappresentazioni mentali (mental rappresentation )} 
	
	\begin{itemize}
		\item \textbf{Entità}: Sono gli elementi che compongono il mondo: mela, casa, persona, ..... 
			
		\item \textbf{Fatti}:
		Sono delle proprietà che descrivono le entità, queste proprietà possono essere vere o false : la mela è rossa, il cavallo è blu mondo". Possono essere time invariant o no (realative al tempo ) o space invariant 
	\end{itemize}
	
	
	\subsection{Rappresentazioni Mentali}
	
	Come suggerisce il nome la rappresentazione mentale non è altro come noi rappresentiamo il mondo nella nostra testa. E la differenza fra il mondo e la nostra prappresentazione e quel \textit{Semantic gap} di cui abbiamo parlato prima.
	
		\underline{Volevo inserire un immagine qui ma non me lo fa fare :(}
	
	
	
	Ci sono 2 tipi di rappresentazioni del mondo :
	\begin{itemize}
		
		\item \textbf{Analogic mental rappresentation}:
		Rappresenta il mondo come noi lo vediamo (il primo impatto se si può dire) non filtrato da nessun tipo di linguaggio.  Es: Stai camminando vedi un auto non pensi "Ho visto un auto" tu vedi l'auto nessun filtro "informazione pura". 
		Sono un insiemi di preceti che si trasformano in fatti.
		
		Questa rappre sentazione ci permette di acquisire conscenza dal mondo 
		
		 \item \textbf{Linguistic mental rappresentation}:
		 Questa invece è generata dal linguaggio . Sono rappresentazioni che descrivono le Analogic mental rappresentation attraverso il linguaggio.
		 
		 Usa le  \textbf{sentenze} che sono formate attraveso l'utilizzo di un alfabeto e regole di una \textbf{lingua}.
		 
		 Ci permettono di esprimere concetti più astratti : Java, Oriented object programing (l'hai mai visto nel mondo?)
		 
\[
\text{Precetti (dalle analogical mental representations)} \longrightarrow{} \text{Concetti (attraverso l'alfabeto)} \longrightarrow{} \text{Sentenze (regole sintattiche di formazione) che descrivono cosa vogliamo rappresentare (conoscenza).}
\]
		 
	\end{itemize}

	\underline{Volevo inserire un immagine qui ma non me lo fa fare :(}\\
	
	
	La consocenza di una persona è l'unione di queste 2 rappresentazioni\\
	
	
	{\large Problema}\\

	Per via del \textit{Semantica gap} ogni rappresentazione mentale differà da quelle delle altre persone quindi ci sono \textbf{infinite rappresentazioni mentali !}. Differenze nelle analogic mental rappresentation conduce a differenze nelle linguistic mental rappresentation.
	
	Le mental rappresentation sono diverse per via delle differenti \textbf{cordinate spazio temporali} in cui sono state generate (io adesso, io fra 20 anni) e per lo \textbf{scopo} (il pilota di un aereo e l'igeniere che lo costruisce hanno 2 rappresentazioni diverse ).
	
	
	Oltre a differire alcune rappresentazioni possono essere \textbf{incoerenti}:

	Una rappresentazione mentale è incoerenti quando rappresenta uno stato del mondo che è impossibile per come lo conosciamo. Coerenza significa assenza di incoerenza.
	
	2 mental rappresentation possono essere \textbf{Mutualmente incosistenti} : 
	solo 1 di loro può essere vera allo stesso tempo (oggi alle 4 sono a Roma, oggi alle 4 sono a Mosca). Nel caso contrario in cui 2 mental rappresentation non sono mutualmente incosistenti diciamo che sono \textbf{compatibili}
	
	
	Come abbiamo detto in precedenza le rappresentazioni differiscono (sono soggettive) e quindi 2 differenti mental rappresentation della stessa cosa possono esse \textit{mutualmente inconsistenti} , La presenza di incoerenza fornisce la prova della soggettività
	delle rappresentazioni mentali coinvolte
\end{document}