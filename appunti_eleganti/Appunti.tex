\documentclass [12pt, letterpaper]{article}
\usepackage{amsmath}


\begin{document}
	\title{Appunti di Logica computazionale}
	\author{Rovesti Alberto}
	\date{2024}
	
	\maketitle
	
	Lo scopo di questa materia è quello di formalizzare il processo logico definendo verità oggettive in modo da creare ragionamenti/ conclusioni non ambigui
	
	
	
	\section{Concetti base}
	\begin{itemize}
		\item \textbf{Percezione} :
		
		Attraverso i nostri sensi noi non percepiamo la totalità del mondo e intlre possiamo percepirlo distorto Es: illusioni ottiche, giochi di luce 
		
		\item \textbf{Concettualizzazione} :
		
		Una volta osservato il mondo ci costruiamo in mente dei concetti che descrivono questo mondo: es:
		ho visto 100 sedie , se vedio un altra sedia anche diversa da tutte quelle viste prima la riconosco perchè ho una rappresentazione concettuale di cosa una sedia sia. Questo porta ad errori perchè come noi concepiamo le cose può essere influenzato dal nostro linguaggio (es: esperimento schianto macchine - a delle persone è stato fatto vedere un incidente e dopo gli hanno chiesto cosa fosse successo all'auto, le persone che conducevano gli esperimenti hanno usato diverse parole con diverse paersone per porre la domanda (le macchine si sono colpite, si sono scontrate, si sono toccate, si sono schiantate) in base alla gravità del termine utilizzato la risposta era conseguentemente catastrofica )
		
		\item \textbf{Rappresentazione} :
		
		Quando vogliamo descrivere qualcosa a qualcuno commettiamo sempre degli errori di rappresentazione. 
		Esempio: ci ricordiamo male, abbiamo dei baias, memorie parziale e più in generale tralasciamo un infinità di dettagli (che non possiamo comunicare per via del fatto che la nostra memoria è limitata e per un vincolo linguistico)
		
		\item \textbf{Ragionamento} :
		
		Partendo che i nostri pensieri sono basati su questi 3 concetti precedenti la nostra mente ragiona ed elabora e può portare a conclusioni sbagliate 
		
	\end{itemize}
	
	In conclusione il punto chiave è che tutti noi percepiamo il mondo in maniera diversa (semantic gap) e quindi dobbiamo trovare/costruire un sistema che ci permetta di esprimere in maniera inequivocabile se qualcosa sia VERO o FALSO per poter poi costruire concetti più complessi
	
	\section{Elementi fondanti}
	
	Prima di procedere dobbiamo chiarire 2 distinti elementi che sono fondamentali per la costruzione della logica:
	le \textbf{entità (entity)}, i \textbf{fatti (facts)} e le \textbf{Rappresentazioni mentali (mental rappresentation )} 
	
	\begin{itemize}
		\item \textbf{Entità}: Sono gli elementi che compongono il mondo: mela, casa, persona, ..... 
			
		\item \textbf{Fatti}:
		Sono delle proprietà che descrivono le entità, queste proprietà possono essere vere o false : la mela è rossa, il cavallo è blu mondo". Possono essere time invariant o no (realative al tempo ) o space invariant 
	\end{itemize}
	
	
	\subsection{Rappresentazioni Mentali}
	
	Come suggerisce il nome la rappresentazione mentale non è altro come noi rappresentiamo il mondo nella nostra testa. E la differenza fra il mondo e la nostra prappresentazione e quel \textit{Semantic gap} di cui abbiamo parlato prima.
	
		\underline{Volevo inserire un immagine qui ma non me lo fa fare :(}
	
	
	
	Ci sono 2 tipi di rappresentazioni del mondo :
	\begin{itemize}
		
		\item \textbf{Analogic mental rappresentation}:
		Rappresenta il mondo come noi lo vediamo (il primo impatto se si può dire) non filtrato da nessun tipo di linguaggio.  Es: Stai camminando vedi un auto non pensi "Ho visto un auto" tu vedi l'auto nessun filtro "informazione pura". 
		Sono un insiemi di preceti che si trasformano in fatti.
		
		Questa rappre sentazione ci permette di acquisire conscenza dal mondo 
		
		 \item \textbf{Linguistic mental rappresentation}:
		 Questa invece è generata dal linguaggio . Sono rappresentazioni che descrivono le Analogic mental rappresentation attraverso il linguaggio.
		 
		 Usa le  \textbf{sentenze} che sono formate attraveso l'utilizzo di un alfabeto e regole di una \textbf{lingua}.
		 
		 Ci permettono di esprimere concetti più astratti : Java, Oriented object programing (l'hai mai visto nel mondo?)
		 
\[
\text{Precetti (dalle analogical mental representations)} \longrightarrow{} \text{Concetti (attraverso l'alfabeto)} \longrightarrow{} \text{Sentenze (regole sintattiche di formazione) che descrivono cosa vogliamo rappresentare (conoscenza).}
\]
		 
	\end{itemize}

	\underline{Volevo inserire un immagine qui ma non me lo fa fare :(}\\
	
	
	La consocenza di una persona è l'unione di queste 2 rappresentazioni\\
	
	
	{\large Problema}\\

	Per via del \textit{Semantica gap} ogni rappresentazione mentale differà da quelle delle altre persone quindi ci sono \textbf{infinite rappresentazioni mentali !}. Differenze nelle analogic mental rappresentation conduce a differenze nelle linguistic mental rappresentation.
	
	Le mental rappresentation sono diverse per via delle differenti \textbf{cordinate spazio temporali} in cui sono state generate (io adesso, io fra 20 anni) e per lo \textbf{scopo} (il pilota di un aereo e l'igeniere che lo costruisce hanno 2 rappresentazioni diverse ).
	
	
	Oltre a differire alcune rappresentazioni possono essere \textbf{incoerenti}:

	Una rappresentazione mentale è incoerenti quando rappresenta uno stato del mondo che è impossibile per come lo conosciamo. Coerenza significa assenza di incoerenza.
	
	2 mental rappresentation possono essere \textbf{Mutualmente incosistenti} : 
	solo 1 di loro può essere vera allo stesso tempo (oggi alle 4 sono a Roma, oggi alle 4 sono a Mosca). Nel caso contrario in cui 2 mental rappresentation non sono mutualmente incosistenti diciamo che sono \textbf{compatibili}
	
	
	Come abbiamo detto in precedenza le rappresentazioni differiscono (sono soggettive) e quindi 2 differenti mental rappresentation della stessa cosa possono esse \textit{mutualmente inconsistenti} , La presenza di incoerenza fornisce la prova della soggettività
	delle rappresentazioni mentali coinvolte
	
	
	\section{Rappresentation}
	
	Le rappresentazioni sono una parte del mondo creato dagli uomini che rappresenta il la mental rappresentation dell'uomo e sono accessibili tramite i 5 sensi. Esempio: un diesgno di una schimmia rappresenta la nostra concezione dei scimmia e lo puoi vedere attraverso gli occhi 
	
	\underline{\textit{Inserire immaggine 5.png }}
	
	Le rappresentazioni possono essere di 2 tipi :
	
	\begin{itemize}
		\item \textbf{Analogical Representations}: 
			Raffigurano le \textit{analogic mental rappresentation}. Es : foto, video, audio
			
		\item \textbf{Linguistic Representations}: 
		Desctivono le \textit{analogic mental rappresentation}. Es : testo, documento ,...		
	\end{itemize}

Naturalmente le Analogical Representations sono innate per natura se vedi un disegno nessuno ti ddeve spiegare cos'è. Mentre le Linguistic Representations sono apprese es: devi prima imparare a leggere 
\\

Le rappresentazioni sono costruite con l'obiettivo di minimizzare la probabilità di diverse interpretazioni e, quindi, di mental rappresentation.



\section{World model}

Per rappresentare il mondo in maniera efficace utilizziamo questo set di regole:
Il world model deve codificare:
\begin{enumerate}
	\item Le informazioni codificate nel analogical representations
	\item Le informazioni codificate nel linguistic representations
	\item la mappatura fra lingustic e mental rappresentation
\end{enumerate}

Tutti e tre i componenti sopra menzionati devono essere codificati in modo intuitivo e univoco.


In pratica si traduce con :

\begin{itemize}
	\item Percetti e parole che li concettualizzano (AR -> LR)
	\item Percetti organizzati in fatti (AR)
	\item Frasi che li descrivono (LR)
	\item Frasi mappate in fatti tramite concetti (LR->AR)
\end{itemize}


\subsection{Set teory}
	per rappresentare il nostro world model usiamo la \textbf{set teory}. Perchè ? rappresentazione non ambigua, formalizza le entità, le loro somiglianze e dissomiglanze ed è universale
	
	
	Ecco come rappresentiamo gli elementi nella set teroy : 
	
	\subsection{Precept}
	Un precetto è qualcosa che viene percepito, ovvero un insieme di proprietà che distinguono un oggetto o un evento da altri. Un percept è ciò che il cervello riconosce come unità separata dal resto, basandosi su una combinazione di proprietà. 
	
	
	\underline{\textit{Non ho capito bene questa parte, bisogna finirla}}
	
	
	\subsection{Facts e Models}
	Un fatto è una dichiarazione di realtà che coinvolge i percetti, ma che ha anche una componente spazio-temporale. In altre parole, un fatto descrive una situazione concreta che si verifica in un determinato punto nello spazio e nel tempo.
	Tipi di facts:
	
	\begin{itemize}
		\item \textbf{Spacetime invariant facts}:  Io sono un umano
		\item \textbf{Time invariant facts}: la terra è nella via lattea
		\item \textbf{Space invariant facts}: Se vgolio atteraversare l'oceano devo nuotare o volare 
		\item \textbf{Spacetime variant facts}: Io sono amico con Zeno (ancora per poco però)
	\end{itemize}
	
	I percetti sono le unità di base della percezione (proprietà, entità, relazioni), mentre i fatti sono dichiarazioni che combinano questi percetti con un riferimento specifico a quando e dove accadono
	
	
	\subsection{Model}
	Un modello \textbf{M} è un insieme di fatti (atomici)
		\begin{center}
		$ M = {f} $
		\end{center}
	Esempio : \{Sofia is a person, Paolo is a man,Rocky is a dog,
		Sofia is near Paolo,Sofia has blond hair, Sofia is a friend of Paolo,
		Rocky is an animal, Rocky is the dog of Sofia, ...\} 
		
		
		Per essere più precisi :
	
	\textbf{Definizione (Domain of interpretation).} A Domain (of interpretation) èun insieme di fatti $\{f\}$.
	
	\[
	D = \{f\}
	\]
	

	
	\textbf{Definizione (Model).} Dato un domain $D$, un model $M$ e un sottoinsieme di  $D$.
	
	\[
	M = \{f\} \subseteq D
	\]
		
		

Un domain è l'insieme di tutti i fatti che siamo disposti a considerare. Un model è solo il sottoinsieme di fatti che definiamo come raffiguranti la situazione attuale

Un domain è tutto quello tutto quello che si può potenzialmente percepire \textbf{anche fatti inconsistenti tra loro}. Sono poi i modelli che non devono essere incosistenti. Un dominio è solamente un insieme che ragruppa tutti i fatti . Poi il modello sceglie quali "usare" per la situazione attuale 


\underline{\textit{Di nuovo non ho capito bene lo schema qua bisogna riguarfarlo }}

\subsection{Assertions}
Un'\textbf{asserzione} $a$ è una frase \textbf{atomica}, cioè una frase che non può essere scomposta in frasi più semplici, che descrive in modo univoco un singolo fatto. Es : Mario è un uomo	
\\

\textbf{Assertional theories} are linguistic descriptions of models\\ \\
	\textbf{Definizione (Assertional theory).} Un Assertional theory $T_{A}$, è un insieme di asserzioni  

\[
T_{A} = \{a\} 
\]

\subsection{Language}
	Una lingua è qualsiasi notazione
	(alfabeto + regole di formazione che generano frasi) definita
	dagli esseri umani, concordata dagli esseri umani, che consente di
	descrivere rappresentazioni analogiche.
	
	
	\textbf{Definizione  (Assertional language).} Un assertional language $L_A$ è un insieme di aserzioni $\{a\}$.
	
	\[
	L_A = \{a\}
	\]
	
	\vspace{1cm}
	
	\textbf{Definizione (Assertional theory).} Data un assertional language $L_A$, un assertional theory $T_A$ è un sottoinsieme di  $L_A$.
	
	\[
	T_A = \{a\} \subseteq L_A
	\]
\end{document}